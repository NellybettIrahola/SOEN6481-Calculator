%%%%%%%%%%%%%%%%%%%%%%%%%%%%%%%%%%%%%%%%%
% Twenty Seconds Resume/CV
% LaTeX Template
% Version 1.1 (8/1/17)
%
% This template has been downloaded from:
% http://www.LaTeXTemplates.com
%
% Original author:
% Carmine Spagnuolo (cspagnuolo@unisa.it) with major modifications by 
% Vel (vel@LaTeXTemplates.com)
%
% License:
% The MIT License (see included LICENSE file)
%
%%%%%%%%%%%%%%%%%%%%%%%%%%%%%%%%%%%%%%%%%

%----------------------------------------------------------------------------------------
%	PACKAGES AND OTHER DOCUMENT CONFIGURATIONS
%----------------------------------------------------------------------------------------

\documentclass[../main.tex]{subfiles} % a4paper for A4

%----------------------------------------------------------------------------------------
%	 PERSONAL INFORMATION
%----------------------------------------------------------------------------------------

% If you don't need one or more of the below, just remove the content leaving the command, e.g. \cvnumberphone{}

\profilepic{images/profilePro.png} % Profile picture

\cvname{James Brown} % Your name
\cvjobtitle{Ph.D. Mathematics} % Job title/career

\cvdate {Gender: Male} % Date of birth
\cvaddress{Age: 40-65 years} % Short address/location, use \newline if more than 1 line is required
\cvnumberphone{Location: Montreal, Canada} % Phone number
\cvsite{University: Concordia University} % Personal website
\cvmail{Email: james.brown@concordia.ca} % Email address

%----------------------------------------------------------------------------------------

\begin{document}
\begin{changemargin}{+5cm}{+5.5cm}
%----------------------------------------------------------------------------------------
%	 ABOUT ME
%----------------------------------------------------------------------------------------

\aboutme{} % To have no About Me section, just remove all the text and leave \aboutme{}

%----------------------------------------------------------------------------------------
%	 SKILLS
%----------------------------------------------------------------------------------------

%----------------------------------------------------------------------------------------

\makeprofiles % Print the sidebar

%----------------------------------------------------------------------------------------
%	 INTERESTS
%----------------------------------------------------------------------------------------

\parte{About Me}

James Brown is a professor of Mathematics at Concordia University. He has always been interested in mathematics and he thinks Number Theory is the purest mathematics that could exist, so he followed through undergraduate and graduate school. \newline
%----------------------------------------------------------------------------------------
%	 EDUCATION
%----------------------------------------------------------------------------------------

\parte{Business Goal}

He wants to continue as a professor in Concordia University. He thinks a calculator can not contribute to his current work since there are better software and languages that he can recommend to his students and use for his research.\newline


\parte{Experience \& Skills}

He is an expert in Number Theory \& Computational Algebra. He is personally interested in Algebraic Number Theory as research topic. He has skills with several programming languages and tools. He uses different systems for his work and projects such as SageMath and Wolfram and recommend its use to his students.\newline

%----------------------------------------------------------------------------------------
%	 USER REQUIREMENTS
%----------------------------------------------------------------------------------------

\parte{User Requirements}

\begin{enumerate}
   \item Open source calculator.
   \item The calculator should provide information of the different ways of classifying a number and operations with them.
   \item The calculator should be available online.
   \item The calculator should provide information about the multiples ways of representing the number.
   \item The calculator should identify the position of where certain patterns of number occur.
    \newline  
\end{enumerate}

%----------------------------------------------------------------------------------------
%	 OTHER INFORMATION
%----------------------------------------------------------------------------------------

\parte{Other Information}

He said that he has never used the number before. He questions the uses of the constant; in his opinion it is a made-up number that one expects would be transcendental and have the properties of transcendental numbers, but he doesn’t see other uses. He also said that he would not use a calculator with this number since there are many other tools. \newline


\end{changemargin}
\end{document} 
