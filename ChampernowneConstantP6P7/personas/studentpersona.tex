%%%%%%%%%%%%%%%%%%%%%%%%%%%%%%%%%%%%%%%%%
% Twenty Seconds Resume/CV
% LaTeX Template
% Version 1.1 (8/1/17)
%
% This template has been downloaded from:
% http://www.LaTeXTemplates.com
%
% Original author:
% Carmine Spagnuolo (cspagnuolo@unisa.it) with major modifications by 
% Vel (vel@LaTeXTemplates.com)
%
% License:
% The MIT License (see included LICENSE file)
%
%%%%%%%%%%%%%%%%%%%%%%%%%%%%%%%%%%%%%%%%%

%----------------------------------------------------------------------------------------
%	PACKAGES AND OTHER DOCUMENT CONFIGURATIONS
%----------------------------------------------------------------------------------------

\documentclass[../main.tex]{subfiles} % a4paper for A4

%----------------------------------------------------------------------------------------
%	 PERSONAL INFORMATION
%----------------------------------------------------------------------------------------

% If you don't need one or more of the below, just remove the content leaving the command, e.g. \cvnumberphone{}

\profilepic{images/profile.png} % Profile picture

\cvname{David Wilson} % Your name
\cvjobtitle{Math Student} % Job title/career

\cvdate {Gender: Male} % Date of birth
\cvaddress{Age: 20-30 years} % Short address/location, use \newline if more than 1 line is required
\cvnumberphone{Location: Caracas, Venezuela} % Phone number
\cvsite{University: Universidad Simon Bolivar (USB)} % Personal website
\cvmail{Email: david.wilson@usb.ve} % Email address

%----------------------------------------------------------------------------------------

\begin{document}
\begin{changemargin}{+5cm}{+5.5cm}
%----------------------------------------------------------------------------------------
%	 ABOUT ME
%----------------------------------------------------------------------------------------

\aboutme{} % To have no About Me section, just remove all the text and leave \aboutme{}

%----------------------------------------------------------------------------------------
%	 SKILLS
%----------------------------------------------------------------------------------------

%----------------------------------------------------------------------------------------

\makeprofiles % Print the sidebar

%----------------------------------------------------------------------------------------
%	 INTERESTS
%----------------------------------------------------------------------------------------

\parte{About Me}

David Wilson is a Venezuelan student. He is studying for a Master degree in Mathematics at Universidad Simon Bolivar. He has always been interested in mathematics and passionate about numbers and integrals. He is also interested in technology and uses several systems to do his work. \newline

\parte{Business Goal}

He expects to finish his studies and become a professor. He would like to do research in one of his passions (numbers, integrals). He thinks a calculator can contribute to that goal and become an important tool for his current and future work.\newline

%----------------------------------------------------------------------------------------
%	 EDUCATION
%----------------------------------------------------------------------------------------

\parte{Experience \& Skills}

He is an expert in Number Theory \& Computational Algebra. However, he is personally interested in Approximation Theory and Probability Theory as research topics. He has skills with several programming languages such as R, Python, and others. He also uses different systems for his work and projects.\newline

%----------------------------------------------------------------------------------------
%	 PUBLICATIONS
%----------------------------------------------------------------------------------------

\parte{User Requirements}

\begin{enumerate}
 
   \item Open source calculator.
   \item Calculator that can be accessed without using the Internet.
   \item The calculator should recognize patterns and associate them with its translation, like a dictionary.
   \item The calculator should decrypt and encrypt messages.
   \item The calculator should provide information about other numbers.
   \item The calculator should provide generic operations.
    \newline  
\end{enumerate}

%----------------------------------------------------------------------------------------
%	 AWARDS
%----------------------------------------------------------------------------------------

\parte{Other Information}

For him the main applications of the Champernowne Constant is related with cryptography specifically encryption, this number in conjunction with other numbers can be used to encode and decode any type of message. \newline

\end{changemargin}
\end{document} 
